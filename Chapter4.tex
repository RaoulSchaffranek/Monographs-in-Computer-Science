\documentclass{article}[12pt]
\usepackage{amsmath}
\usepackage{amsfonts}
\usepackage{tikz}
\title{Focus Tutorial Chapter 4 Exercises}
\author{Raoul Schaffranek}
\date{September 11, 2018}

\newcommand{\concat}{\ensuremath{^{\frown}}}
\newcommand{\stuterringremoval}{\ensuremath{\propto}}
\newcommand{\cons}{\ \&\ }
\newcommand{\stream}[1]{\langle#1\rangle}
\newcommand{\naturals}{\ensuremath{\mathbb{N}}}
\newcommand{\prefixordering}{\ensuremath{\sqsubseteq}}
\newcommand{\length}{\ensuremath{\#}}
\newcommand{\filter}{\ \ensuremath{\text{\textcircled{s}}}\ }
\newcommand{\rt}{\ensuremath{\textbf{rt}}}
\newcommand{\tm}{\ensuremath{\textbf{tm}}}
\newcommand{\ts}{\ensuremath{\textbf{ts}}}
\newcommand{\rng}{\ensuremath{\textbf{rng}}}
\newcommand{\dom}{\ensuremath{\textbf{dom}}}
\newcommand{\map}{\ensuremath{\textbf{map}}}
\begin{document}
   \maketitle
   \section{Exercise 4.1}
   For which stream $s$ do we have that $\dom.s = \{\}$?
   \paragraph{Solution}{
       Let $s \in M^{\omega}$ be an untimed steam.

       Then $\dom.s = \{\}$ if and only if $s = \stream{}$.

       By contradiction, assume $\exists m \in M. s(0) = m$

       then $m \in \dom.s$, thus $\dom.s \neq \{\}$

       Now, let $s \in M^{\underline{\omega}}$ be a timed stream.

       Then $\dom.s = \{\}$ if and only if $s = \stream{}$.

       By contradiction, assume $\exists m \in M \cup \{\checkmark\} . s(0) = m$

       then $m \in \dom.s$, thus $\dom.s \neq \{\}$.
   }
   \section{Exercise 4.2}
   If $s \concat r = r$ for streams $s$ and $r$, what can we conclude about $s$ and $r$?
   \paragraph{Solution}{
       Let $s, r \in M^{\omega}$ be untimed streams.

       \begin{enumerate}
           \item{
            We observe that $s = \stream{}$ satsfies the proposition for every $r \in M^{\omega}$, becaucse $s \concat r = \stream{} \concat r = r$.
           }
           \item{
            If the proposition holds and $s \in M^{*}$ then $\forall n \in \naturals . s^n \prefixordering r$. 
            We know that $s \concat r \prefixordering r$ because $s \concat r = r$ and $\prefixordering$ is reflexive.
            We can generalize this observation to iterated streams of $s$ like follows:
            $\forall n \in \naturals . {s^n} \concat r \prefixordering r$ and therefore $\forall n \in \naturals . {s^n} \prefixordering r$
           }
           \item{
            Furthermore, we observe that if the proposition holds, we know that $s \notin M^{\infty}$ and $s \neq r$ cannot hold in conjunction.

            By the definition of $\concat$ we have $s \concat r = s$ for all $r \in M^{\omega}$ and $r \neq s$ by assumption.
           }
           \item {
               Corollary: $s \in M^{\infty} \Rightarrow s = r$.
           }
       \end{enumerate}

       The same observations can be made for timed streams, with the analagous proves.

   }
   \section{Exercise 4.3}
   What is the length of the stream $\stream{1,2}^{3}$?

   \paragraph{Solution}{
       $\length \stream{1,2}^3 = \length(\stream{1,2} \concat \stream{1,2} \concat \stream{1,2}) = \length \stream{1,2,1,2,1,2} = 6$.
    }
   \section{Exercise 4.4}
   Which stream $s \in \naturals^{\omega}$ fulfills the equation $s = 1 \cons s$?
   \paragraph{Solution}{
       The stream $s \in \naturals^{\infty}$ with $\dom.s = \{1\}$.
       We show that $s$ is the only fixed point of the equation.
       First, observe that $s$ is indeed a fixed point, because $s \cons 1 = s$ holds.
       Moreover, $s$ is the only fixed point. By contraction, assume there was another
       fixed point $s'$ with $s' \neq s$. Then $s'$ is either finite, in which case
       $s'$ does not satisfy $s' = 1 \cons s'$. Or $s$ and $s'$ disagree on some message, i.e $\exists n \in \naturals . s'(n) \neq s(n) = 1$. Again, $s'$ does not satisfy the equation.
   }
   \section{Exercise 4.5}
   What can we conclude about a stream $s$ for which $\rt.s = s$?
   \paragraph{Solution}{
       Either $s = \stream{}$ or $s \in M^{\infty}$ and $\exists m \in M.\forall n \in \naturals.s(n) = m$.
       For the first case, we have $\rt.\stream{} = \stream{}$ by definition.
       Now, assume $s \neq \stream{}$.
       If $s$ was finite, then clearly $\length \rt.s \neq \length s$, hence $\rt.s \neq s$.
       Thus, if $s$ is not empty it must be infinite. Moreover, the domain of $s$ must be a singelton-set.
       Assume otherwise, i.e. $\exists m. \land s(m) \neq s(m+1)$. For every such $m$ we than have $\rt.s(m+1) \neq s(m+1)$,
       thus $\rt.s \neq s$.
   }
   \section{Exercise 4.6}
   Prove that the following propositions holds.\\
   $r \prefixordering s \Rightarrow t \concat r \prefixordering t \concat s$
   \paragraph{Solution}{
       By induction on the length of $t$.
       If $t = \stream{}$, we have $\stream{} \concat r \prefixordering \stream{} \concat s$
       and thus $r \prefixordering s$ because $\stream{}$ is left-neutral with respect to $\concat$.
       If $\length t = \infty$, we have $t \concat r \prefixordering t \concat s$ which simplifies to
       $t \prefixordering t$, which holds because $\prefixordering$ is reflexive.
       Now, assume that $r \prefixordering s \Rightarrow t \concat r \prefixordering t \concat s$ holds
       for streams $t$ of some fixed length. We need to show $(m \cons t) \concat r \prefixordering (m \cons t) \concat s$. By definition of $\cons$ we obtain $(\stream{m} \concat t) \concat r \prefixordering (\stream{m} \concat t) \concat s$, and because $\concat$ is associative we further have $\stream{m} \concat (t \concat r) \prefixordering \stream{m} \concat (t \concat s)$. We can now apply the indution hypothesis proving the proposition.
   }
   \section{Exercise 4.7}
   A function $f \in M^{\omega} \rightarrow M^{\omega}$ on streams is called prefix monotonic, if for all streams $r$ and $s$: $r \prefixordering s \Rightarrow f(r) \prefixordering f(s)$\\
   Show that the operator $\rt$ is prefix monotonic.
    \paragraph{Solution} Assume the premise $r \prefixordering s$ holds. We show the conclusion by induction over the lenght of $s$.
    \begin{itemize}
        \item {Case: $s = \stream{}$.\\
            Then, $r = \stream{}$ because its the only prefix of $s$.\\
            Thus, $\rt.r = \stream{} \prefixordering \rt.s$ follows directly, because the empty stream is a prefix of every stream.
        }
        \item { Case: $s = m \cons t$ for some message $m$ and stream $t$.\\
            Then, $\rt.s = t$. By induction over the length of $r$.
            \begin{itemize}
                \item { Case: $r = \stream{}$.\\
                    Then $\rt.r = \stream{} \prefixordering \rt.s$, because the empty stream is a prefix of every stream.
                }
                \item { Case: $r = n \cons u$ for some message $n$ and stream $u$.\\
                    Then, $\rt.r = u$.
                    Thus, $r \prefixordering s \Rightarrow u \prefixordering t \Rightarrow \rt.r \prefixordering \rt.s$.
                }
            \end{itemize}
        }
    \end{itemize}
   \section{Exercise 4.8}
   Given the stream $r$ such that $r = \stream{1,2} \concat r$
   \begin{enumerate}
        \item prove that $\length r = \infty$
        \item describe the streams denoted by: $\{1\} \filter r$, $r|_{5}$, $\stuterringremoval r$, $\map(r, g)$\\
        where $g \in \naturals \rightarrow \naturals$ is defined as $g(n) = \text{ if } odd(n) \text{ then } 1 \text{ else } 2 \text{ fi }$
   \end{enumerate}
    \paragraph{Solution 1}{
        Assume, $\length r \neq \infty$. Then, $\length(\stream{1,2} \concat r) = 2 + r \neq r$.
        Thus, $r \ \neq \stream{1,2} \concat r$.
    }
    \paragraph{Solution 2}{
        \begin{itemize}
            \item $\{1\} \filter r$ is a possibly infinite stream with $\rng.s = \{1\}$.
            \item $r|_5$ is a stream with $\length r = 5$ and $r|_5 \prefixordering r$.
            \item $\stuterringremoval r$ is a stream with $\rng.(\stuterringremoval r) = \rng.r$,
                $\dom.(\stuterringremoval r) = \dom.r$ and $\forall m \in \naturals. m < \length r \Rightarrow s(m) \neq s(m+1)$.
            \item $\map(r,g)$ is a stream with $\forall n \in (\dom.r) . \map(r,g)(n) = g(n)$ and $\length r = \length \map(r,g)$. In natural language: if $m$ is the message of $r$ at position $n$, then the stream $\map(r,g)$ has a corresponding message at position $n$, that is $1$ if $m$ is odd and otherwise $2$. Moreover, both streams  have the same length.
        \end{itemize}
    }

   \section{Exercise 4.9}
   Given the timed stream $r$ such that $r = \stream{1,\checkmark,2}\concat r$\\
   describe the elements denoty by: $\overline{r}, r\downarrow_{5}$, $\tm(r,5)$, $\ts(r)$
   \paragraph{Solution}
   \begin{enumerate}
        \item $\overline{r}$ is the unique solution to the equation: $r' = \stream{1,2} \concat r'$.
            In natural language: $\overline{r}$ is the untimed stream, that is like $r$ but with ticks $\checkmark$ removed, or in otherwords: $r$ alternates forever between messages $1$ and $2$.
        \item $r\downarrow_{5}$ is $\stream{1,5,\checkmark,1,5,\checkmark,1,5,\checkmark,1,5,\checkmark,1,5,\checkmark}$
        \item $\tm(r,5)$ is $3$
        \item $\ts(r)$ is $False$ because $r$ is not time-synchrnonous.
   \end{enumerate}
   \section{Exercise 4.10}
   Given the stream $r$ such that $r = \checkmark \cons r$\\
   prove that $\overline{r} = \stream{}$
   \paragraph{Solution}{
       By contradiction, assume that $\overline{r} \neq \stream{}$.
       Then $\overline{r}(0) = m$ for some $m \in \rng.r$ with $m \neq \checkmark$.
       But then, the assumption $r = \checkmark \cons r$ is violated.
    }
   \section{Exercise 4.11}
   Which streams $r \in M^{\underline{\infty}}$ fulfill the equation\\
   $\overline{r} = \overline{\checkmark \cons r}$
   \paragraph{Solution}{
        Every stream $r$ fulfills the equation.
        $\overline{\checkmark \cons r} = M \filter (\checkmark \cons r) = M \filter (\stream{\checkmark} \concat r) = (M \filter \stream{\checkmark}) \concat (M \filter r) = \stream{} \concat (M \filter r) = M \filter r = \overline{r}$.
   }
\end{document}
